\epigraph{The law that entropy always increases, holds, I think, the supreme position among the laws of Nature. If someone points out to you that your pet theory of the universe is in disagreement with Maxwell's equations — then so much the worse for Maxwell's equations. If it is found to be contradicted by observation — well, these experimentalists do bungle things sometimes. But if your theory is found to be against the second law of thermodynamics I can give you no hope; there is nothing for it but to collapse in deepest humiliation.}{\textbf{Arthur Eddington \\ -- The nature of the physical world (1928)}}


\section{The Smoluchowski equation} \label{Smoluchowski}
Here we would like to explain our model for Brownian dynamics with self-induced temperature gradients. If we know the temperature as a function of space and time, then we can use the Smoluchowski equation (equation \ref{eqn:smoluchowskiIntro}) to model the motion. One can interpret the relation between equation \ref{eqn:smoluchowskiIntro} and Figures \ref{fig:Schematic}, \ref{fig:landauersBlowtorch} as follows: Brownian particles are subject to a given potential and are agitated by local thermal noise. These agitations give the particles the energy required to move over barriers in the potential. As one could imagine, these thermal interactions draw energy from the environment in the form of heat. One may imagine that these interactions could cause the temperature of the environment to change. Normally two simplifying (not always explicit) assumptions are made at this point \cite{Reimann2001}. (i) That the temperature gradients created by the motor are very small compared to the thermal energy of the surrounding environment which is assumed to be effectively infinite and (ii) that when these temperature fluctuations do occur, they diffuse away so rapidly that they do not need to be accounted for. In this project, we will question the second assumption and explore the consequences of not making this assumption.

First, we should derive the heat produced by the Brownian particle as it evolves on a potential, which is done in Ref \cite{M.W.Jack2016}. Here we will reproduce the derivation because it is very important. Notice that the potential energy of the motor constrained to a domain $\Omega$ is given by $U = \int_\Omega P(x, \, t) V(x) dx$. Differentiating with respect to time, we get
\begin{eqnarray}
\frac{d U}{d t} &=& -\int_\Omega \frac{\partial J(x, \, t)}{\partial x} V(x) dx \\
		        &=& -[J(x) V(x)]_{\partial \Omega} + \int_\Omega J(x, \, t) \partial_x V(x, \, t) dx.
\end{eqnarray}
The first term can be interpreted as the work done on the particle at the boundaries while the integrand in the second term is the local heat flux produced by the particle which we will write as $\dot{q}(x, \, t)$. Now that we know the heat being produced by the particle, we are ready to quantify the self-induced temperature gradients. First of all, note that if there are no sources of heat, then the temperature will evolve according to the equation
\begin{equation}
\frac{\partial T(x, \, t)}{\partial t} = D\frac{\partial^2 T(x, \, t)}{\partial x^2}, \label{eqn:heatNoSource}
\end{equation} 
which is the heat equation for an incompressible stationary fluid. $D$ is the thermal diffusivity of the environment, throughout this project we will consider $D$ to be constant. Equation \ref{eqn:heatNoSource} will cause any temperature gradients to diffuse with time. As we have mentioned, the movement of the Brownian particle acts as a source of heat, by adding this heat to equation \ref{eqn:heatNoSource} we get
\begin{equation}
\frac{\partial T(x, \, t)}{\partial t} = -\kappa \dot{q}(x, \, t) + D \frac{\partial^2 T(x, \, t)}{\partial x^2}. \label{eqn:tmpHeat}
\end{equation}

%Assumption (ii) has also been questioned previously by Streater in the context of Brownian motion \cite{Streater1997, Streater1997a}. In these articles, Streater investigates Brownian motion from a microscopic view and then comes up with a mathematical model to describe Brownian particles that are thermally coupled to the environment, he then goes on to prove that the model is thermostatistically consistent in the sense that energy is conserved and that entropy increases. We will explore a similar set of equations in the context of Brownian motors and we will try to determine the length scales at which the thermal interaction is important.

The combination of equations \ref{eqn:smoluchowskiIntro} and \ref{eqn:tmpHeat} represent the physical coupling of temperature gradients to Brownian motion. The majority of the effort expended in this thesis will be in understanding the nature of these two coupled partial differential equations. Let us reiterate these equations here so that they can be understood in full and so that they can be referred to later. The system is written as
\begin{eqnarray}
J(x, \, t) &=& -\gamma^{-1} \left [ \frac{\partial V(x)}{\partial x} P(x, \, t) + k_B T(x, \, t) \frac{\partial P(x, \, t)}{\partial x} \right ], \label{eqn:current} \\
\frac{\partial P(x, \, t)}{\partial t} &=& -\frac{\partial J}{\partial x}, \label{eqn:Smoluchowski} \\
\frac{\partial T(x, \, t)}{\partial t} &=& -\kappa \dot{q}(x, \, t) + D \frac{\partial^2 T(x, 
\, t)}{\partial x^2}, \label{eqn:TemperatureEvolution}
\end{eqnarray}

where
\begin{itemize}
\item{$P(x, \, t)$ is the probability density as a function of  reaction coordinate $x$ and time $t$}
\item{$J(x, \, t)$ is the probability current}
\item{$\gamma$ is the friction coefficient}
\item{$V(x)$ is the potential for the motor}
\item{$k_B$ is the Boltzmann constant}
\item{$\dot{q}(x, \, t) = \partial_x V(x, \, t) J(x, \, t)$ is the local heat from the motor}
\item{$\kappa$ dictates how much the motor effects the environment. In ~\autoref{thermodynamics} we will show that $\kappa$ is one over the specific heat of the environment}
\item{$D$ is the thermal diffusivity}
\end{itemize}
Equation \ref{eqn:Smoluchowski} is called the Smoluschowski equation which we noted earlier and equation \ref{eqn:TemperatureEvolution} is the heat equation. These equations make our intuitive notions more precise, we see that the first term on the right hand side of the Smoluchowski equation (equation \ref{eqn:Smoluchowski}) is a drift term that is forced by our potential and that the second term contains a diffusion term that is scaled by our temperature.  Coupled models of this type have been mentioned earlier by Streater \cite{Streater1997, Streater1997a,Streater2000,Streater1997b}. Our work uses a similar model that was modified to suit our needs.

Intuitively, the first term in equation \ref{eqn:TemperatureEvolution} will create temperature gradients and the second term will cause these temperature gradients to flatten out. If the first term dominates, then the temperature gradients will grow to a very large size. In this case, we have to concede that our model will no longer be relevant. In particular our model does not include phase changes, so at the very least we require that the temperature gradients do not cause the fluid that the particle is in to freeze. This can be done if one makes sure that the second term is not significantly smaller than the first. In ~\autoref{dimensionless} we will show that this puts a restriction on the characteristic energy of the potential. On the other hand, if the second term completely dominates, then any temperature gradients created by the Brownian particle will immediately diffuse away. In this regime, equation \ref{eqn:TemperatureEvolution} will not have an effect on the evolution of the system and the model will be reduced to the Smoluchowski equation by itself. The regime that we are interested in is the regime where the first and second terms are both important. In this regime, the Brownian particle will have an effect on the environment but this effect will not cause the system to behave in a way that causes our model to break down.

One may notice that equations \ref{eqn:Smoluchowski} and \ref{eqn:TemperatureEvolution} are a non-linear system of equations, if one looks at the last term of equation \ref{eqn:current}, then one will find that the second spatial derivative of $P(x, \, t)$ is multiplied by the temperature, which, due to equation \ref{eqn:TemperatureEvolution} is a function of $P(x, \, t)$. Because of this, equation \ref{eqn:current} is a second order non-linear equation. Likewise if one looks at the first term of equation \ref{eqn:TemperatureEvolution}, then one will find that the temperature is multiplied by the first derivative of the probability distribution (which is a function of temperature), therefore equation \ref{eqn:TemperatureEvolution} is also non-linear. Because we are dealing with non-linear equations, we must take extra care when finding their solutions numerically. In ~\autoref{numerics} we will discuss the techniques used to solve the equations as well as analytical and convergence tests that we used to test the performance of these numerical techniques.

In fluid dynamics, equation \ref{eqn:TemperatureEvolution} is called the heat equation, in our case we have neglected the term that represents the flow of the fluid that the Brownian particle is situated in. Including this flow would involve adding $u(x) \frac{\partial T}{\partial x}$ to the left hand side of equation \ref{eqn:TemperatureEvolution}, where $u(x)$ is velocity of the flow of the fluid. The role of fluid dynamics is discussed further in ~\autoref{fluidDynamics}. Here we will note that the assumption that the fluid is stationary ($u(x) = 0$) means that the temperature gradients cannot be allowed to become too large because temperature gradients are known to create flows in fluids. Even though these flows may effect the validity of our model in real life, here we will neglect them for three main reasons. (i) They are not related to the physics that we would like to capture in this project since this is a project on statistical mechanics, not fluid dynamics. (ii) We would not like to depart too far from the explored literature, which is similar to reason (i), however here we would like to stress that our model is just an extension of previous models. Extending the extension before understanding the extension could be premature. (iii) Including the flow into our equations of motion would increase the complexity of the system beyond the scope of an honors project: although simply adding the fluid flow to equation \ref{eqn:TemperatureEvolution} may seem like an innocent operation, calculating the way in which the temperature gradients and the flow of the Brownian particle effect the fluid flow would mean adding a third equation to our system of equations.

Our model includes a temperature that depends on $x$ and $t$, which has been explored in the literature \cite{Kramers1940}. Some authors write the current (equation \ref{eqn:current}) as \cite{Kramers1940,Kampen1988}
\begin{equation}
J(x, \, t) = -\gamma^{-1} \frac{\partial}{\partial x} \left (\frac{\partial V(x, \, t)}{\partial x} P(x, \, t) + k_B \frac{\partial}{\partial x}[T(x, \, t) P(x, \, t)] \right ),  \label{eqn:diffCurrent}
\end{equation}
the difference being that they put the temperature inside the inner derivative. We will return to this difference in ~\autoref{thermodynamics}, where we will discuss the role that the current has in the generation of entropy.
Our model departs from previous work in that the temperature now depends on the evolution of the probability distribution as well.

A restriction that we must enforce on our model so that equations \ref{eqn:Smoluchowski} and \ref{eqn:TemperatureEvolution} are valid is that there is only a very small number of Brownian particles. This is because the temperature gradients can act as inter-particle interactions. To see this, imagine that there are two particles moving about and creating temperature gradients. If one particle moves to a location where the other has created a temperature gradient, then it will be affected by the temperature gradient created by the other particle. Therefore, to model this, one would need to model the probability distributions of each particle separately and make sure that the interactions are accounted for. We stress this point because often people will use the Smoluchowski equation to describe the density of particles in a fluid. Interpreting $P(x, \, t)$ in this way in the presence of self induced temperature gradients would neglect the inter-particle interactions and would not be an accurate description, therefore we will be careful to make sure that $P(x, \, t)$ is used to model the probability distribution of a single colloidal particle, not many.  


Temperature gradients that are self induced in Brownian dynamics have been known for some time. A particularly interesting example is the Soret-Dufour effect \cite{Onsager1931,HortLinzLuecke1992,PiazzaGuarino2002,Santamaria-HolekGadomskiRubi2011}. In this effect, the diffusion of the concentration of a species of molecules causes a measurable effect on the temperature. Another interesting example is the reverse Landauer blowtorch \cite{DasDasBarikEtAl2015}, where a nett force displacing a Brownian particle can create a temperature gradient. This contrasts the forward Landauer blowtorch where a temperature gradient creates a psuedo-force on the particles. We will explore the reverse Landauer blowtorch in \autoref{reverseLandauerBlowtorch} where we discuss how our model adds to previous work.

\section{Boundary conditions}
Equations \ref{eqn:Smoluchowski} and \ref{eqn:TemperatureEvolution} model a system that is embedded in a much larger environment. In order to talk about solutions to these equations, one must define how the system interacts with the environment at the boundaries. In this project, we will discuss the following types of boundary conditions.

\begin{itemize}
\item{Dirichlet: The value of the solution is specified at the boundaries}
\item{Neumann: The value of the first derivative of the solution is specified at the boundaries}
\item{Periodic: The value at the boundaries is not specified, but the left and right boundaries must take on the same value}
\end{itemize}
Each type of boundary conditions comes along with its own physical interpretation, first we will deal with the boundary conditions imposed on the probability distribution.

If the potential goes to infinity at the boundaries, then the probability distribution must vanish at the boundaries. On the other hand, there are some potentials that are periodic where the value of the potential is the same finite number at both sides. In this case it can be interesting to apply periodic boundary conditions to the probability distribution. Physically this corresponds to a system where a particle passing through the left side will appear at the right side and vice-versa. One can picture this by imagining a particle constrained to a circular domain. The other type of boundary condition that we impose on the probability density is that the value of the density is zero at the boundaries. Mathematically, this is a special case of Dirichlet boundary conditions, physically this corresponds to so called reflecting boundary conditions \cite{Gardiner2009}. Reflecting boundary conditions correspond to the physical situation where a particle is strongly constrained to a region, this could occur if a barrier that the particle cannot pass through is placed at the boundaries on either side of the domain. One can also achieve reflecting boundary conditions by making the potential go to infinity at the boundaries.

As for the temperature, both Neumann and Dirichlet boundary conditions are realized. In the case of Neumann boundary conditions, the derivative is set to zero at both boundaries. This is a very interesting case to imagine, because the derivative of the temperature physically represents the flow of heat in our system, therefore the physical interpretation of a vanishing derivative at the boundary is that no heat is flowing through the boundaries, or in other words the system is enclosed in a perfectly insulating box. If in addition there is no probability flowing out of the boundaries then energy is conserved and Neumann boundary conditions correspond to a closed system that is completely unaffected by the rest of the universe.

The second type of boundary condition that can be imposed on the temperature is the Dirichlet type where the temperature is held fixed at both boundaries. The physical interpretation of this requirement is that the domain is embedded in a much larger system that is at a fixed temperature. The tacit assumption made is that no matter what the system does, it is not able to affect the temperature of the environment. Since heat is flowing through the boundaries, the local energy of the system is not conserved. In \autoref{thermodynamics} we will quantify the energy flowing through the boundaries and show that the first law of thermodynamics is upheld.
% THERMODYNAMICS
\section{System thermodynamics} \label{thermodynamics}
Equation \ref{eqn:Smoluchowski} and equation \ref{eqn:TemperatureEvolution} define equations of motion for our system, the system may be confined to a region $\Omega$ embedded in a larger environment which interacts with our system through the boundary conditions. In this section, we will show that our equations of motion obey the first and second laws of thermodynamics.

In the over-damped limit, the kinetic energy of the Brownian particle is negligible, therefore the energy of the system is completely described by the potential energy of the particle plus the thermal energy of the bath. The potential energy of the particle is $U_P = \int_{\Omega} V(x) P(x) dx$ and if the fluid that we are in is incompressible, then the thermal energy of the bath is $c_p \int_{\Omega} T(x) dx$, where $c_p$ is the specific heat capacity of the environment. To see that this is the thermal energy consider the units of the quantity $c_p T(x, \, t)$. These are J K$^{-1}$ K, so $c_p T(x, \, t) dx$ is the heat content of an infinitesimal element $dx$. With this we have the total energy of the system as
\begin{equation}
E(t) = \int_{\Omega} V(x)P(x, t) dx + c_p \int_{\Omega} T(x, t) dx. \label{eqn:energy}
\end{equation}
By using equations \ref{eqn:Smoluchowski} and \ref{eqn:TemperatureEvolution}, we can differentiate the energy with respect to time to get
\begin{align}
\frac{d E}{d t} & = \int_{\Omega} V(x) \frac{\partial P}{\partial t} dx + c_p \int_{\Omega} \frac{\partial T}{\partial t} dx \\
 & = -\int_{\Omega} V(x) \frac{\partial J}{\partial x} + c_p \int_{\Omega} -\kappa J(x) \frac{\partial V}{\partial x} + D \frac{\partial^2 T}{\partial x^2} dx \\
 & = [V(x)J(x)]_{\partial \Omega}+ \int_{\Omega} \frac{\partial V}{\partial x} J(x) dx - \kappa c_p \int_{\Omega} \frac{\partial V}{\partial x} J(x) dx + c_p D \left [\frac{\partial T}{\partial x} \right]_{\partial \Omega}.
\end{align}
Notice that the middle two terms need to cancel for the change in energy to equal the flow of energy through the boundaries. Therefore, the first law of thermodynamics requires that $\kappa = \frac{1}{c_p}$. Physically we can understand this by looking at the first term of equation \ref{eqn:TemperatureEvolution}. When the heat capacity is small, $\kappa$ becomes large and in this case, even a small amount of heat being produced by the Brownian particle will have a large effect on the evolution of the temperature. Conversely, if the heat capacity is large, then the heat produced by the Brownian particle will have a very small effect on the temperature. We therefore expect that equation \ref{eqn:TemperatureEvolution} can be neglected in the case where the environment has a very large heat capacity compared to the heat being produced by the Brownian particle.

For the entropy of the total system, we have \cite{Streater1997a}
\begin{equation}
S(t) = -k_B \int_{\Omega} P(x, t) \log(P(x, \, t)) dx + c_p \int_{\Omega} \log(T(x, \, t)) dx.
\end{equation}
The first term is the Gibbs entropy in the continuous case \cite{Jaynes1965} and the second term is the entropy of an incompressible fluid \cite{CengelBoles1994}. Using the entropy of an incompressible fluid restricts the sytems that can be modeled by our equations of motion. In particular, we will not be able to treat Brownian particles suspended in a gas since a gas is compressible by defintion. The goal of this project is not to create a system that models many different situations in nature, but rather to create a system that is self-consistent.

Differentiating the entropy with respect to time and using equations \ref{eqn:Smoluchowski} and \ref{eqn:TemperatureEvolution}, we get

\begin{align}
\frac{d S}{d t} =  k_B \int_{\Omega} \frac{\partial J}{\partial x} + \frac{\partial J}{\partial x} \log P \ dx + c_p \int_{\Omega} \frac{1}{T} \left(-\kappa J \partial_x V + c_p D \frac{\partial^2 T}{\partial x^2} \right) dx \\
                     = k_B \left ( [J]_{\partial \Omega} + [J \log P]_{\partial \Omega} - \int_{\Omega} \frac{J}{P} \frac{\partial P}{\partial x} dx \right) - \int_{\Omega} \frac{J}{T} \frac{\partial V}{\partial x} + c_p D \int_{\Omega} \frac{1}{T} \frac{\partial^2 T}{\partial x^2} dx.
\end{align}
Denoting the boundary terms as $B(t) = k_B( [J \log P]_{\partial \Omega} + \left[\frac{\partial J}{\partial x} \right]_{\partial \Omega} ) $, we can write:

\begin{eqnarray}
\frac{d S}{d t} & = & - \int_{\Omega} k_B \frac{J}{P} \frac{\partial P}{\partial x} + \frac{J}{T} \frac{\partial V}{\partial x} dx +  c_p D \int_{\Omega} \frac{1}{T} \frac{\partial^2 T}{\partial x^2} dx + B(t) \\
                    & = & \gamma \int_{\Omega} \frac{J^2}{T P} dx + c_p D \int_{\Omega} \frac{1}{T} \frac{\partial^2 T}{\partial x^2} dx + B(t), \label{eqn:entropyGen}
\end{eqnarray}
where in the second equality we used the definition of $J$ given by \ref{eqn:current}.

By noticing that
\begin{equation}
\frac{\partial}{\partial x} \left(\frac{1}{T} \frac{\partial T}{\partial x} \right) = -\frac{1}{T^2} \left(\frac{\partial T}{\partial x} \right)^2 + \frac{1}{T} \frac{\partial^2 T}{\partial x^2},
\end{equation}
we can rewrite the second term of equation \ref{eqn:entropyGen} as:
\begin{eqnarray}
c_p D \int_{\Omega} \frac{1}{T} \frac{\partial^2 T}{\partial x^2} dx &=& c_p D \int_{\Omega} \frac{\partial}{\partial x} \left(\frac{1}{T} \frac{\partial T}{\partial x} \right) + \frac{1}{T^2} \left(\frac{\partial T}{\partial x} \right)^2 dx \\
 &=& c_p D \int_{\Omega} \frac{1}{T^2} \left(\frac{\partial T}{\partial x} \right)^2 dx + c_p D\left[\frac{1}{T} \frac{\partial T}{\partial x} \right]_{\partial \Omega}.
\end{eqnarray}
Define
\begin{eqnarray}
\dot{S}_{gen} &\equiv& \int_{\Omega} \gamma \frac{J^2}{T P} + c_p D \frac{1}{T^2} \left(\frac{\partial T}{\partial x} \right)^2 dx \label{eqn:entropyGen} \\
%\text{and} \nonumber \\
%B(t) &\equiv& k_B \left( [J \log P]_{\partial \Omega} + \left[\frac{\partial J}{\partial x}\right]_{\partial \Omega} \right) + c_p D\left[\frac{1}{T} \frac{\partial T}{\partial x} \right]_{\partial \Omega},
\end{eqnarray}
so the change in entropy can be written as:
\begin{equation}
\frac{d S}{d t} = \dot{S}_{gen} + B(t) + c_p D\left[\frac{1}{T} \frac{\partial T}{\partial x} \right]_{\partial \Omega}.
\end{equation}
We notice that the change in entropy is equal to a positive number $\dot{S}_{gen}$ plus the entropy flowing through the boundaries, this is precisely the second law of thermodynamics. Furthermore, the generated entropy in equation \ref{eqn:entropyGen} can be split into two terms. The first term is to be interpreted as the entropy generated by the Brownian particle; since the motion of the particle is random, as the Brownian particle diffuses we will become less certain of its position. The second term is the entropy generated by the diffusion of temperature gradients; if one studies the solutions to the heat equation, then one will find that any spikes in the temperature will flatten out and eventually the temperature will be uniform. This flattening out of the temperature is irreversible and is therefore associated with an increase in entropy. When one discusses heat engines implemented on the molecular scale, one finds that the Carnot efficiency can only be attained when the entropy generation is zero. Simply by reading equation \ref{eqn:entropyGen}, we can see that this occurs when the current is zero and the temperature is flat. In ~\autoref{SteadyState} we will show that this is not possible for a Brownian particle in a tilted potential. 

Some authors write the current in the form of equation \ref{eqn:diffCurrent}, however if we use this definition of $J$, then $\dot{S}_{gen}$ is not necessarily positive, in fact with this term in the equation $\dot{S}_{gen}$ becomes:
\begin{equation}
\dot{S}_{gen} \equiv \int_{\Omega} \gamma \left(\frac{J^2}{T P} - \frac{J}{T} \frac{\partial T}{\partial x} \right) + c_p D \frac{1}{T^2} \left(\frac{\partial T}{\partial x} \right)^2 dx.
\end{equation}
With this version of the current, we can get local decreases in entropy, therefore we do not use this version of the current. Our version of the current is supported by Streater \cite{Streater1997, Streater1997a,Streater2000,Streater1997b} and reduces to the other version in the case of constant temperature.

% DIMENSIONLESS
\section{Making the equations dimensionless}  \label{dimensionless}

Upon viewing equations \ref{eqn:Smoluchowski} and \ref{eqn:TemperatureEvolution}, we see that there is a large number of constants that are set by the properties of the Brownian particle that we are modeling. We would like to reduce the number of variables for two reasons: (i) By reducing the number of variables we will hopefully gain a more concise physical description of the system; and (ii) Having a small number of free variables is very convenient for creating a program to approximate the equations numerically and dimensionless equations tend to be less prone to numerical error because they avoid cases where small numbers are compared to large ones in a floating point system.
Here we will non-dimensionalize the equations. To do this, introduce $\hat{x} = \frac{x}{L}$, where $L$ is the length scale of the system, the Smoluchowski equation now becomes

\begin{equation}
\frac{\partial P}{\partial t} = \gamma^{-1}\frac{1}{L^2} \frac{\partial}{\partial \hat{x}} \left (P \frac{\partial V}{\partial \hat{x}} + k_B T \frac{\partial P}{\partial \hat{x}} \right ).
\end{equation}
Now let $E_0$ be the characteristic energy of the system, for a chemical reaction, $E_0$ will be the $\Delta G$ of the reaction and for a periodic potential, $E_0$ will be amplitude of the oscillations. Now we introduce the dimensionless potential and the dimensionless temperature as $\hat{V}(x) = \frac{V(x)}{E_0}$ and $\hat{T}(x) = \frac{k_B T(x)}{E_0}$ respectively. Now the Smoluchowski equation becomes

\begin{equation}
\frac{\partial P}{\partial t} = \frac{E_0}{\gamma L^2} \frac{\partial}{\partial \hat{x}} \left (P \frac{\partial \hat{V}}{\partial \hat{x}} + \hat{T} \frac{\partial P}{\partial \hat{x}} \right ).
\end{equation}
Let $\hat{t} = \frac{E_0}{\gamma L^2} t$ and $\hat{P} = L P $, with this we have

\begin{equation}
\frac{\partial \hat{P}}{\partial \hat{t}} = \frac{\partial}{\partial \hat{x}} \left (\hat{P} \frac{\partial \hat{V}}{\partial \hat{x}} + \hat{T}  \frac{\partial \hat{P}}{\partial \hat{x}} \right ). \label{eqn:dimensionlessSmoluchowski}
\end{equation}
Defining $\hat{J} = \frac{\gamma L^2}{E_0} J $, applying these definitions to the equation for the temperature evolution, we find that

\begin{equation}
\frac{E_0^2}{\gamma k_B L^2} \frac{\partial \hat{T}}{\partial \hat{t}} = -\kappa \frac{E_0}{\gamma L^2}\hat{J}(\hat{x}) \frac{E_0}{L} \frac{\partial \hat{V}}{\partial \hat{x}} + \frac{D E_0}{k_B L^2} \frac{\partial^2 \hat{T}}{\partial \hat{x}^2}.
\end{equation}
Now let 
\begin{equation}
\alpha = \frac{\kappa k_B}{L} \quad \text{or} \quad \alpha = \frac{k_B}{c_p L} \\
\end{equation}
\begin{equation}
\beta = \frac{D \gamma}{E_0},
\end{equation}
then we have
\begin{equation}
\frac{\partial \hat{T}}{\partial \tau} = -\alpha \hat{J}(\hat{x}) \frac{\partial \hat{V}}{\partial \hat{x}} + \beta \frac{\partial^2 \hat{T}}{\partial \hat{x}^2}. \label{eqn:dimensionlessHeat}
\end{equation}

As for the energy of the system, the dimensioned equation is

\begin{equation}
E(t) = \int P(x) V(x) dx + \frac{1}{\kappa} \int T(x) dx
\end{equation}
Defining $\hat{E}(t) = \frac{E(t)}{E_0}$, we have

\begin{eqnarray}
\hat{E}(t) & = & \int \hat{P} \hat{V} d \hat{x} + \frac{L}{k_B \kappa} \int \hat{T} d\hat{x} \\
              & = & \int \hat{P} \hat{V} d \hat{x} + \frac{1}{\alpha} \int \hat{T} d\hat{x}.
\end{eqnarray}

So our system depends on the parameters $\alpha$ and $\beta$ as well as the shape of the potential. Physically, $\alpha$ represents how much the particle interacts with the local environment and $\beta$ represents how quickly the temperature gradients diffuse. As the length scale of the system increases, $\alpha$ will decrease, therefore the temperature gradients become negligible for large systems. We interpret this as meaning that the temperature gradients are much smaller than the system and can diffuse away quickly. As the characteristic energy of the system increases, $\beta$ will decrease meaning that the diffusion of temperature will be negligible. We interpret this as meaning that for high energy systems, temperature gradients are created much faster than they can be diffused away. From now on when we refer to the dimensionless equations, we will omit the hats for notational convenience.


%In order to neglect the first term of \ref{eqn:dimensionlessHeat}, we need to have $\alpha \hat{J} \hat{V} \ll \frac{\partial \hat{T}}{\partial \tau}$, this means that  we must have:
%\begin{equation}
%\frac{k_B}{c_P L} \frac{\gamma L^2}{E_0} J(x, \, t) \frac{L}{E_0} \partial_x V(x, \, t) \ll \left (\frac{E_0}{\gamma L^2} \right)^{-1} \frac{E_0}{k_B} \frac{\partial T(x, \, t)}{\partial t}
%\end{equation}
%So in terms of the physical properties of the physical parameters of the environment,
%\begin{equation}
%\frac{1}{c_p} \left (\frac{k_B}{E_0} \right)^2 \ll (J(x, \, t) \partial_x V(x, \, t) )^{-1} \frac{\partial T(x, \, t)}{\partial t} \label{eqn:neglectAlpha}
%\end{equation}
%We can understand this physically as follows: $E_0$ represents the typical barrier height that the Brownian particle will encounter in it's energy landscape. When these barriers are large compared to the energy of the particle, the particle will essentially be trapped and the temperature gradients that it produces will be negligible. Also, when the specific heat of the environment is large, any heat that the particle draws from the bat will have a negligible effect on the actual temperature of the environment.

%Meanwhile, in order for us to neglect the second term, we need to have:
%\begin{equation}
%\frac{D \gamma}{E_0} L^2 \frac{\partial^2 \hat{T}(x, \, t)}{\partial x^2} \ll \left (\frac{E_0}{\gamma L^2} \right)^{-1} \frac{\partial \hat{T}(x, \, t)}{\partial t}
%\end{equation}
%This means that
%\begin{equation}
%D \ll 1  \label{eqn:neglectBeta}
%\end{equation}
%or in other words, the length scale of the system is much larger than the thermal diffusivity. Physically this means that if there are any initial thermal gradients, then they will remain in place over time because the thermal diffusion is too slow to dissipate them. Interestingly this regime does not necessarily mean that temperature gradients should be neglected entirely, this is because $\beta$ is independent of $\alpha$. If we meet the condition in equation \ref{eqn:neglectBeta}, but at the same time we do not meet the condition in equation \ref{eqn:neglectAlpha}, then the Brownian particle will create temperature gradients that will not diffuse away rapidly. In these cases, the small thermal gradients may diverge to arbitrary sizes as time goes to infinity. Our model is only valid for small deviations in temperature, in particular our model does not consider any phase changes so the temperature gradients must be constrained so that they do not cause the fluid to freeze. We will therefore avoid these situations in our numerical exploration of the system.

%There is one more condition that can be met under which the system effectively reduces to the Smoluchowski equation, imagine that we have a system that begins with a constant temperature in space. As time goes on, the term involving $\alpha$ induces temperature gradients and the term involving $\beta$ will cause these temperature gradients to flatten out. If the time scale of the thermal diffusion is much faster than the time scale of the production of thermal gradients, then the temperature gradients will all diffuse faster than they can be created, we can describe this mathematically by:
%\begin{equation}
%\alpha \hat{J} \partial_{\bar{x}} \hat{V} \ll \beta \frac{\partial \hat{T}(\bar{x}, \, t)}{\partial \bar{x}}
%\end{equation}
%we can reduce this down to the following condition:
%\begin{equation}
%\frac{k_B}{c_p E_0 D} \ll 1
%\end{equation}
%If this condition is true, then any temperature gradients that arise will diffuse away very quickly meaning that our model will reduce to the Smoluchowski equation.

%So far we have shown the conditions that allow us to either neglect specific terms in equation \ref{eqn:dimensionlessHeat}, but now we would like to discuss those conditions under which we need to use this equation in order to properly model a system. Currently we can say when certain terms can be neglected, but we would like to be able to determine when they are important. Specifically, we would like to know the conditions when the both terms are non-negligible and when the temperature gradients do not rapidly diffuse away. This means that all of the above conditions must be violated, we will write this as
%\begin{eqnarray}
%\frac{E_0}{L} &\sim& \frac{k_B}{c_p} \\
%L &\sim& D \\
%E_0 &\sim& \frac{k_B}{c_p D}
%\end{eqnarray}
%this is an upper bound for both $E_0$ and $L$ as well as an upper bound for their ratio.
%
%At standard temperature and pressure ($T = 298.15 K$ and pressure $P_{atm} = 101.3 kPa$), water has the following properties:
%\begin{itemize}
%\item{Specific heat: 4186 $J K^{-1}$}
%\item{Thermal diffusivity: $1.43 \times 10^{-5} m^2 s^{-1}$}
%\end{itemize}
%we also note that as per usual the Boltzmann constant is given by $k_B = 1.381 \times 10^{-23} J K^{-1}$. Typical chemical reactions that occur in biological molecular motors have energy on the order of $k_B T \approx 3.451 \times 10^{-22} J$ {\color{red} [citations]}. The upper bound for our energy is $E_{\text{upper}} \sim \frac{k_B}{c_p D} \approx \frac{1.381 \times 10^{-23}}{4186 \times 1.43 \times 10^{-5}} \approx 2.31 \time 10^{-22} J$. This means that molecular motors operate at the energy scale where equation \ref{eqn:dimensionlessHeat} is important, as for length scale 

\section{A brief discussion of the model}
We are now in a position that we may discuss the physics of the equations of motion in some detail. In particular we would like to discuss the assumptions that have to be made in order for the equations to be valid. These assumptions have been explained sporadically throughout the text and we would like to compile them into a short list that the reader should keep in mind throughout the remainder of the thesis.
\begin{enumerate}
\item{The degrees of freedom of the bath operate at a time scale much faster than that of the Brownian particle itself}
\item{The kinetic energy of the particle is negligible (over-damped limit)}
\item{There is only one Brownian particle in our model}
\item{The fluid that the particle is in is incompressible}
\item{The fluid that the particle is in is not flowing}
\item{The temperature gradients are small enough that they do not induce a phase change}
\end{enumerate}
We can take any of these phenomena into account by making changes to our equations of motion, however these small changes to the equation of motion can cause the system to become very complex. Having said that, some of the assumptions made turn out to be more reasonable than one may have thought on first sight. Assumptions 1 and 2 are assumptions that are made when deriving the Smoluchowski equation from the Langevin equation, these assumptions have been discussed at length and turn out to be very reasonable \cite{Reimann2001,Gardiner2009,Einstein1905,KellerBustamante2000,Kramers1940}. Meanwhile, assumption 3 means that we cannot talk about the very interesting cooperative effects that occur in motors such as myosin, however this assumption is fine for motors such as kinesin or for colloidal particles in a potential, both of which are very interesting single particle phenomena \cite{Leibler1993}. Assumption 5 is perfectly fine in water which is the fluid of choice for all biological molecular motors. Assumption 6 is fine as long as we do not have significantly large temperature gradients. Finally assumption 7 just puts a restriction on the parameters that we choose in our simulations. As far as we are aware, there are no experimentally realized examples of Brownian particles creating a phase change in the fluid that they are suspended in.





