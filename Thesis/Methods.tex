In order to see how these equations behave with time, we have to resort to numerical methods (see section \ref{numerics}). However we note that for a given potential there will be a stationary solution that we will refer to as the ``steady state'', given periodic boundary conditions, we will derive an analytical form for this steady state.

\section{Steady state solution} \label{SteadyState}
In the steady state, we have:

\begin{eqnarray}
\frac{\partial P(x, t)}{\partial t} &=&  0 \ = \ \frac{\partial J}{\partial x} \label{eqn:SmoluchowskiSteady} \\
\frac{\partial T(x, t)}{\partial t} &=& 0 \ = \ -\kappa q(x,t) + \frac{\partial}{\partial x} \left ( D \frac{\partial T(x, t)}{\partial x} \right ) \label{eqn:TemperatureSteady}
\end{eqnarray}

Suppose that we have the following boundary conditions:
\begin{align}
P(x = 0) & = P(x = L) \\
J(x = 0) & = J(x = L) \\
\left. \frac{\partial T}{\partial x} \right \rvert_{x = 0} & = 0 = \left. \frac{\partial T}{\partial x} \right \rvert_{x = L} \label{eqn:temperatureBoundary}
\end{align}
where $L$ is the length scale of the system. Physically, these conditions say that the nett current flowing out of the boundaries is zero and that no heat escapes from the system, thus the energy of the sytem is conserved. Section 5.2 of \cite{Gardiner2009} gives the steady state current as:

\begin{equation}
J_s = \left [\frac{2 k_B T(L)}{\psi(L)} - \frac{2 k_B T(0)}{\psi(0)}  \right] P_s(0) \left [\int_0^L dx'/\psi(x') \right]^{-1}
\label{eqn:SteadyCurrent}
\end{equation}

with $\psi(x) \equiv \exp[-\int_0^x dx' \frac{\partial_x V(x')}{2 k_B T(x')}]$. Meanwhile, the density is:

\begin{equation}
P_s(x) = P_s(0) \left [\frac{\int_0^x \frac{dx'}{\psi(x')} \frac{T(L)}{\psi(L)} + \int_x^L \frac{dx'}{\psi(x')} \frac{T(0)}{\psi(0)} }{\frac{T(x)}{\psi(x)} \int_0^L \frac{dx'}{\psi(x')} } \right]
\label{eqn:SteadyDensity}
\end{equation}
In this case, $J_s$ is a constant and $P_s(0)$ is also a constant. Assuming that we know these constants it is now possible to find the steady state temperature. We have:

\begin{equation}
\frac{\partial T}{\partial t} = 0 = -\kappa J_s \partial_x V + D \frac{\partial^2 T}{\partial x^2} \label{eqn:steadyTemperatureODE}
\end{equation}

In one dimension, \ref{eqn:steadyTemperatureODE} can be written as an ordinary differential equation of the form
\begin{equation}
T''(x) = \frac{\kappa J_s}{D} V'(x)
\end{equation}

We can solve this equation by integrating both sides twice to give:

\begin{equation}
T(x) = \frac{\kappa J_s}{D} \int_0^x V(x') dx' + \xi x + d \label{eqn:steadyTemperature}
\end{equation}
for unkown constants $\xi$ and $d$. By applying the boundary condition \ref{eqn:temperatureBoundary}, we find that
\begin{align}
T'(0) & = 0 = \frac{\kappa J_s}{D} V(0) + \xi \\
T'(L) & = 0 = \frac{\kappa J_s}{D} V(L) + \xi
\end{align}
This implies that either $J_s = 0$ or $V(0) = V(L)$, meaning that the coupled system does not admit a steady state solution for a tilted potential. Later, we will see that it is possible to have a steady state in higher dimensions where the flow of heat from the environment can dissipate the heat produced by the Brownian particle. By recalling that $E = \int_0^L P(x') V(x') dx' + c \int_0^L T(x') dx'$ we are able to find an expression for $d$. First, we will integrate the temperature from $0$ to $L$
\begin{equation}
\int_0^L T(x') dx' = \frac{\kappa J_s}{D} \int_0^L dx' \int_0^{x'} V(x'') dx'' + \xi \frac{L^2}{2} + d L
\end{equation}
Therefore,
\begin{equation}
	d = \frac{1}{L} \left(c E - c\int_0^L P(x') V(x') dx' - \frac{\kappa J_s}{D} \int_0^L dx' \int_0^{x'} V(x'') d x'' + \xi \frac{L^2}{2} \right)
\end{equation}

It would seem that one should be able to calculate the steady state current and density directly from the equations shown above. However, we notice that the constants $J_s$ and $P_s(0)$ have to satisfy equations (\ref{eqn:SteadyCurrent}), (\ref{eqn:SteadyDensity}) and (\ref{eqn:steadyTemperature}) while also satisfying the normalization condition $\int_0^L P(x) dx = 1$. To do this we define an objective function given by

\begin{equation}
obj(J_s, P_s(0)) = \left (J_s - \left [\frac{2 k_B T(L)}{\psi(L)} - \frac{2 k_B T(0)}{\psi(0)}  \right] P_s(0) \left [\int_0^L dx'/\psi(x') \right]^{-1} \right)^2  \label{eqn:Objective}
\end{equation}

And we minimize this objective function with respect to $J_s$ and $P_s(0)$ under the constraint $\int_0^L P(x) dx = 1$. Another way to do this is to guess a steady state density and temperature and use finite differencing to simulate forward in time until the transients die out.

%----------------------------------------------------------------------------------------
%	NUMERICS
%----------------------------------------------------------------------------------------

\section{Numerical simulation} \label{numerics}

\subsection{Dimensionalizing the equations}

%In order to make the numerical simulation easier we will dimensionalize the equations, to do this, introduce $\bar{x} = \frac{x}{L}$, then the Smoluchowski equation becomes
%
%\begin{equation}
%\frac{\partial P}{\partial t} = \gamma^{-1}\frac{1}{L^2} \frac{\partial}{\partial \bar{x}} \left (P \frac{\partial V}{\partial \bar{x}} + k_B \frac{\partial}{\partial \bar{x}}(TP) \right )
%\end{equation}
%Now let $E_0$ be the potential energy gained by moving along one period, i.e. $E_0 = V(x + L) - V(x) = f L$. Now we will introduce the dimensionless potential and the dimensionless temperature as $\hat{V}(x) = \frac{V(x)}{E_0}$ and $\hat{T}(x) = \frac{k_B T(x)}{E_0}$ respectively. Now the Smoluchowski equation becomes
%\begin{equation}
%\frac{\partial P}{\partial t} = \frac{E_0}{\gamma L^2} \frac{\partial}{\partial \bar{x}} \left (P \frac{\partial \hat{V}}{\partial \bar{x}} + \frac{\partial}{\partial \bar{x}}(\hat{T}P) \right )
%\end{equation}
%Let $\tau = \frac{E_0}{\gamma L^2}$, with this we have:
%\begin{equation}
%\frac{\partial P}{\partial \tau} = \frac{\partial}{\partial \bar{x}} \left (P \frac{\partial \hat{V}}{\partial \bar{x}} + \frac{\partial}{\partial \bar{x}}(\hat{T}P) \right )
%\end{equation}
%Applying these definitions to the equation for the temperature evolution, we find that:
%
%\begin{equation}
%\frac{E_0^2}{\gamma k_B L^2} \frac{\partial \hat{T}}{\partial \tau} = -\kappa \frac{E_0}{L^2}\hat{J}(\bar{x}) \frac{E_0}{L} \frac{\partial \hat{V}}{\partial \bar{x}} + \frac{D E_0}{k_B L^2} \frac{\partial^2 \hat{T}}{\partial \bar{x}^2}
%\end{equation}
%Now let $\alpha = \kappa \gamma k_B$ and $\beta = \frac{D \gamma}{E_0}$, then we have
%\begin{equation}
%\frac{\partial \hat{T}}{\partial \tau} = -\alpha \hat{J}(\bar{x}) \frac{\partial \hat{V}}{\partial \bar{x}} + \beta \frac{\partial^2 \hat{T}}{\partial \bar{x}^2}
%\end{equation}
%
%The energy of the system is given by $E = \int P(x) V(x) dx + c \int T(x) dx $, in the dimensionless form

Here we will non-dimensionalize the equations, to do this, introduce $\bar{x} = \frac{x}{L}$, then the Smoluchowski equation becomes

\begin{equation}
\frac{\partial P}{\partial t} = \gamma^{-1}\frac{1}{L^2} \frac{\partial}{\partial \bar{x}} \left (P \frac{\partial V}{\partial \bar{x}} + k_B \frac{\partial}{\partial \bar{x}}(TP) \right )
\end{equation}
Now let $E_0$ be the potential energy difference along one period, i.e. $E_0 = \max(v_0(x)) - \min(v_0(x))$. Now we will introduce the dimensionless potential and the dimensionless temperature as $\hat{V}(x) = \frac{V(x)}{E_0}$ and $\hat{T}(x) = \frac{k_B T(x)}{E_0}$ respectively. Now the Smoluchowski equation becomes

\begin{equation}
\frac{\partial P}{\partial t} = \frac{E_0}{\gamma L^2} \frac{\partial}{\partial \bar{x}} \left (P \frac{\partial \hat{V}}{\partial \bar{x}} + \frac{\partial}{\partial \bar{x}}(\hat{T}P) \right )
\end{equation}
Let $\tau = \frac{E_0}{\gamma L^2}$ and $\hat{P} = L P $, with this we have:

\begin{equation}
\frac{\partial \hat{P}}{\partial \tau} = \frac{\partial}{\partial \bar{x}} \left (\hat{P} \frac{\partial \hat{V}}{\partial \bar{x}} + \frac{\partial}{\partial \bar{x}}(\hat{T} \hat{P}) \right ) \label{eqn:dimensionlessSmoluchowski}
\end{equation}
Now we define $\hat{J} = \frac{\gamma L^2}{E_0} J $, applying these definitions to the equation for the temperature evolution, we find that:

\begin{equation}
\frac{E_0^2}{\gamma k_B L^2} \frac{\partial \hat{T}}{\partial \tau} = -\kappa \frac{E_0}{\gamma L^2}\hat{J}(\bar{x}) \frac{E_0}{L} \frac{\partial \hat{V}}{\partial \bar{x}} + \frac{D E_0}{k_B L^2} \frac{\partial^2 \hat{T}}{\partial \bar{x}^2}
\end{equation}
Now let $\alpha = \frac{\kappa k_B}{L}$ and $\beta = \frac{D \gamma}{E_0}$, then we have

\begin{equation}
\frac{\partial \hat{T}}{\partial \tau} = -\alpha \hat{J}(\bar{x}) \frac{\partial \hat{V}}{\partial \bar{x}} + \beta \frac{\partial^2 \hat{T}}{\partial \bar{x}^2} \label{eqn:dimensionlessHeat}
\end{equation}

As for the energy of the system, the dimensioned version is:

\begin{equation}
E(t) = \int P(x) V(x) dx + \frac{1}{\kappa} \int T(x) dx
\end{equation}
Let $\hat{E}(t) = \frac{E(t)}{E_0}$, we have:

\begin{eqnarray}
\hat{E}(t) & = & \int \hat{P} \hat{V} d \bar{x} + \frac{L}{k_B \kappa} \int \hat{T} d\bar{x} \\
              & = & \int \hat{P} \hat{V} d \bar{x} + \frac{1}{\alpha} \int \hat{T} d\bar{x}
\end{eqnarray}


So our system depends on the parameters $\alpha$ and $\beta$ as well as the shape of the potential. Physically, $\alpha$ represents how much the motor will interact with the environment thermally and $\beta$ represents how quickly the temperature diffuses.

\subsection{Finite differences}
The one dimensional equation can be solved on a discrete grid by using the finite differences method, the main idea behind this strategy is to approximate derivatives with equations of the form:

\begin{equation}
\frac{d f}{d x} \approx \frac{f(x - h) - f(x + h)}{2h}
\end{equation}

for some small $h$. In our simulations, we will use the Crank Nicolson scheme to solve the equations. From now on, we will use the notation that $F(j \Delta x, n \Delta t) = F_j^n$, the key equation for the Crank Nicolson scheme is:
\begin{equation}
\frac{P_j^{n+1} - P_j^n}{\Delta t} = \frac{1}{2}(F_j^{n+1} + F_j^n)
\end{equation}
where $F$ represents the right hand side of the equation that we are doing finite differences on. By applying finite differences to the dimensionless Smoluchowski equation (eq \ref{eqn:dimensionlessSmoluchowski}), we find that:
\begin{equation}
F_j^{i} = \frac{P^i_{j+1} \partial V^i_{j+1} - P^i_{j-1} \partial V^i_{j-1}}{2 \Delta x} + \frac{P^i_{j+1} T^i_{j+1} - 2P^i_j T^i_j + P^i_{j-1} T^i_{j-1}}{\Delta x^2} 
\end{equation}
We make the following definitions:
\begin{align*}
a_j^{n+1} &= -\frac{\partial_x V^{n+1}_{j-1}}{2\Delta x} + \frac{T^{n+1}_{j-1}}{\Delta x^2} \\
b_j^{n+1} &= \frac{-2 T^{n+1}_j}{\Delta x^2} \\
c_j^{n+1} &=  \frac{\partial_x V^{n+1}_{j+1}}{2\Delta x} + \frac{T^{n+1}_{j+1}}{\Delta x^2}  \\
a_j^{n} &= -\frac{\partial_x V^n_{j-1}}{2\Delta x} + \frac{T^n_{j-1}}{\Delta x^2} \\
b_j^{n} &= \frac{-2 T^n_j}{\Delta x^2} \\
c_j^{n} &=  \frac{\partial_x V^n_{j+1}}{2\Delta x} + \frac{T^n_{j+1}}{\Delta x^2}  \\ \numberthis
\end{align*}
With these definitions, the Crank Nicolson scheme can be written down as follows:
\begin{multline}
 -\frac{\Delta t}{2}a_j^{n+1}P_{j-1}^{n+1} + \left (1 - \frac{\Delta t}{2}b_j^{n+1} \right) P_j^{n+1} - \frac{\Delta t}{2} c_j^{n+1} P_{j+1}^{n+1} = a_j^n P_{j-1}^{n}
+ \left (1 + \frac{\Delta t}{2}b_j^n \right) P_j^{n}  + \frac{\Delta t}{2} c_j^n P_{j+1}^{n}
\end{multline}
This equation can be written in matrix form by defining the following matrices:
\begin{equation}
A =
\begin{bmatrix}
	a_0^{n+1} & b_1^{n+1} & 0                & 0                & 0        & \dots                                 & 0        \\
	c_0^{n+1} & a_1^{n+1} & b_2^{n+1} & 0                & 0        & \dots                                 & 0        \\
	0                & c_1^{n+1} & a_2^{n+1} & b_3^{n+1}   & 0        & \dots                                 & 0        \\
			     &                   &                   &                  &           &                                         &           \\
	\vdots         & \vdots         & \ddots         & \ddots         & \ddots & \vdots                              & \vdots \\
			     &                   &                   &                  &           &                                         &           \\
			     &                   &                   &                  &           &                                         &           \\
	0                &                   & \dots           &                   &  c_{J-2}^{n+1} & a_{J-1}^{n+1}  & b_J^{n+1} \\
	0                &                   & \dots           &                   &                         &  c_{J-1}^{n+1} & a_J^{n+1}
\end{bmatrix}
,\quad P^{n+1} =
\begin{bmatrix}
P_0^{n+1}       \\
P_1^{n+1}	     \\
                        \\
                        \\
\vdots               \\
                        \\
                        \\
P_{J-1}^{n+1} \\
P_J^{n+1}
\end{bmatrix}
\end{equation}

\begin{equation}
B =
\begin{bmatrix}
	a_0^{n} & b_1^{n}     & 0                 & 0          & 0                    & \dots            & 0        \\
	c_0^{n} & a_1^{n}     & b_2^{n}      & 0          & 0                    & \dots            & 0        \\
	0                & c_1^{n} & a_2^{n}      & b_3{n} & 0                    & \dots            & 0        \\
			     &               &                   &             &                      &                     &           \\
	\vdots         & \vdots     & \ddots         & \ddots   & \ddots            & \vdots           & \vdots \\
			     &               &                   &             &                      &                     &           \\
			     &               &                   &             &                      &                     &           \\
	0                &               & \dots           &             &  c_{J-2}^{n} & a_{J-1}^{n}  & b_J^{n} \\
	0                &               & \dots           &             &                     &  c_{J-1}^{n} & a_J^{n}
\end{bmatrix}
,\quad P^{n} =
\begin{bmatrix}
P_0^{n}       \\
P_1^n          \\
                    \\
                    \\
\vdots           \\
                    \\
                    \\
P_{J-1}^{n} \\
P_J^{n}
\end{bmatrix}
\end{equation}
With these matrices the equation now becomes $(\mathbb{1} - \frac{\Delta t}{2}A) \cdot P^{n+1} = (\mathbb{1} - \frac{\Delta t}{2}B) \cdot P^n$, this equation can be used to step forward $P$.
Each time that we step forward using this equation we will be out by a factor, this means that at each step we will need to renormalize using the equation $\int P(x) dx = 1$.

Likewise, we can apply the Crank Nicolson scheme to the heat equation, (eq \ref{eqn:dimensionlessHeat}), by looking at the right hand side of this equation, we find that

\begin{equation}
F_j^i = -\alpha \left (P_j^i (\partial_x V_j^i)^2 + \frac{T_{j+1}^i - T_{j-1}^i }{2 \Delta x}\partial_x V_j^i  \right) + \beta \frac{T_{j+1}^i - T_j^i + T_{j-1}^i}{\Delta x^2}
\end{equation}

Just like the discretized Smoluchowski equation, these equations can be written in matrix form. The temperature is normalized by assuming that the energy remains fixed, this will be true as long as no heat or current flows through the boundaries, i.e. $J(x=a) = 0 = J(x = b)$ and $\frac{\partial T}{\partial x} \rvert_a = 0 = \frac{\partial T}{\partial x} \rvert_b$. In this case, the energy is constant and is given by $E = \int P(x) V(x) dx + c_p \int T(x) dx$, so each time that we step the temperature forward, we have to calculate the potential and thermal energy and then scale the temperature so that the total energy remains fixed.

Fortunately the matrices that we are dealing with are very sparse, so the program used to solve these equations can save on memory by calling sparse matrix libraries.

%\subsection{Stochastic methods}
%We can simulate the path of a single particle by using stochastic methods to solve the Langevin equation. By simulating many times we should be able to recover the distributions that were found through finite differencing. When simulating the system  at the microscopic level, equation (\ref{eqn:Smoluchowski}) becomes \cite{Reimann2001}

%\begin{equation}
%\gamma \dot{x}(t) = -V'(x(t)) + \xi(t)
%\end{equation}
%
%Where $\xi(t)$ is called Gaussian white noise of zero mean and has the essential properties that: $\langle \xi(t) \rangle = 0$ and $\langle \xi(t) \xi(s) \rangle = 2 \gamma k_B T \delta(t - s) $. We can also consider the stochastic equation with time discretized, in which case we get:
%
%\begin{equation}
%x(t_{n+1}) = x(t_n) - \Delta t [V'(x(t_n)) + \xi_n]/\gamma
%\end{equation}
%
%This equation can be simulated numerically, or can be used to derive Equation (\ref{eqn:Smoluchowski}) by taking the limit $\Delta t \to 0$ as is done in Appendix B of \cite{Reimann2001}. In the results section, we will show some results of these numerical simulations.
