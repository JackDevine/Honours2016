\chapter{Additional figures}
\renewcommand{\textflush}{flushright} \renewcommand{\sourceflush}{flushright}
\setlength\epigraphwidth{.4\textwidth}
\epigraph{I am a fish.}{\textbf{Philip Brydon -- 2016}}

There are a lot of interesting phenomena that occur when one explores the system that we explored in detail. In this appendix, we would like to share some of the exciting results that we found during the project, these results do not directly contribute to the narrative of the thesis which is why they were not included in the main body of the text. Despite not being a part of the main body, we feel that the reader would be at a loss without being made aware of these phenomena.

\begin{figure}
	\center
	\textbf{Periodic boundary conditions}
	\vspace{0.5cm}

	\begin{subfigure}{0.49\textwidth}
		\includegraphics[width=\textwidth]{PeriodicInit}
	\end{subfigure}
	\begin{subfigure}{0.49\textwidth}
		\includegraphics[width=\textwidth]{PeriodicFinal}
	\end{subfigure}
	\caption{\textbf{An example of a periodic system visualized in one dimension.} We have $\alpha = 1 \cdot 10^{-2} \, m^{-1}$ and $\beta = 1 \cdot 10^{-3}  \, m^2 s^{-1}$}
\end{figure}