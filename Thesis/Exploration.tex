\section{Bistable potentials} \label{Kramers}

\begin{figure}[tb]
\includegraphics[width=\columnwidth]{bistablePotential}
\caption{Bistable potential: In this plot we show the potential where we explore the Kramers rate, the potential has local minima at $a$ and $c$ and a maximum at $b$. If we begin with a probability distribution in the upper well, then the distribution will decay into the ground state of the upper well and then begin to decay into the lower well. The rate of flow from the upper well to the lower one will be denoted by $\kappa_+$ and the rate of flow from the lower well into the upper one will be denoted by $\kappa_-$.}
\label{fig:bistablePotential}
\end{figure}

A bistable potential one that has two stable minima and an intermediate unstable maximum, these potentials occur in a wide range of applications including digital logic \cite{MyersCelebranoKrishnan2015}, protein folding \cite{BryngelsonWolynes1989} and chemical reactions \cite{BernePecora1976}. The bistable potential well is one of the simplest ways to approach the Kramer's rate and many other important properties of a system {\color{red} [citations]}. In the context of Brownian motion, understanding the nature of bistable potentials can help one to build a master equation describing more complicated potentials comprised of multiple deep wells \cite{Barcilon1996, ChallisJack2014}. 

\subsection{Kramer's rate}
Consider the potential shown in Figure \ref{fig:bistablePotential}, if we begin in a state where we are certain that the particle is in the upper well, then as time passes, we should expect the probability distribution to move from point $a$ over the barrier at $b$ and into the well at point $c$. We will consider the regime where $E^+_B = V(x_b) - V(x_a) \gg k_B T$, in this regime the rate at which the particles flow from $a$ to $c$ is given by the Eyring-Kramers law \cite{Eyring1935, Kramers1940}, for our dimensionless equations, this has the form,
\begin{equation}
\kappa_+ = \frac{\sqrt{-V''(x_b) V''(x_a)}}{2 \pi} \exp \left({\frac{-E^+_B}{T}} \right)
\end{equation}
Likewise, there will be a current flowing from $c$ to $a$, we will denote this by $\kappa_-$, once we have calculated both of these rates, the population in the upper well will be given by:
\begin{equation}
P_+(t) = \exp{((\kappa_+ - \kappa_- )t)}.
\end{equation}
We can also achieve this result numerically by starting the system off in the upper well and simulating forward while calculating the probability that the particle is in the upper well at each step. We then fit an exponential to this data and the fitted rate will be our numerically estimated Kramers rate. To assist with measuring the Kramer's rate, we used Hermite interpolation to create a sixth order polynomial with the desired bistable shape. Specifically, we would allow $V(x)$ to be a general 6th order polynomial and then we would specify the value of the polynomial at the locations $a$, $b$ and $c$, we would also enforce that the first derivative vanished at these locations. This would yield 6 equations which we would solve to give us the coefficients of $V(x)$, an example of such a polynomial is shown in Figure \ref{fig:bistablePotential}. Using these interpolating polynomials, we could keep the locations of the wells fixed while controlling the barrier height $E_B^+$. Once we had a potential, we would place the probability distribution in the upper well and then measure the probability that the particle is in the upper well with time and use this information to obtain the Kramer's rate, as shown in Figure {\color{red} make this figure}. This same technique can be used to obtain the Kramer's rate when the temperature is not held fixed. In this case, the Kramer's rate is sensitive to the boundary conditions imposed on the temperature and on the precise values of $\alpha$ and $\beta$ as shown in Figure {\color{red} make this figure}.

\section{The reverse Landauer blowtorch}
As noted in ~\autoref{landauersBlowtorch}, the relative occupancy of wells depends on the spatial distribution of the temperature meaning that the temperature can act as a pseudo-force. However, it has been noted that when the movement of the particle has an effect on the environment, the opposite effect can occur \cite{DasDasBarikEtAl2015}. Concretely, the authors claim that if the probability distribution begins in the upper well, then after they have moved into the lower well they will cause the upper well to become colder in a process called Brownian cooling. In this paper, the authors treat the system stochastically and are not able to model the diffusion of temperature. This is the same as our model with $\beta$ in equation \ref{eqn:dimensionlessHeat} set to zero. Here we will show that this approximation does not always yield an accurate description of the physics involved.

\section{Titlted periodic potentials}
Tilted periodic potentials are very important in biology where they can be used to model molecular motors.