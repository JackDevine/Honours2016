\epigraph{I say that the physics works by pixies, you say that the physics works by gnomes.}{\textbf{Associate Professor Colin Fox -- 2016}}
Brownian dynamics is an area of research that is of interest from the perspective of basic physics as well as applied physics. In this thesis we have discussed Brownian dynamics from a very basic point of view and we have focused in particular on the role of self-induced temperature gradients in the equations of motion. We have created a self-consistent description of Brownian dynamics that fully describes the flow of energy and entropy through a system. The fact that our system has a well defined entropy is a strong advantage and allows one to talk about information in a rigorous way \cite{Landauer1961,MyersCelebranoKrishnan2015}. Moving beyond the analytical solutions, we found that allowing self induced temperature gradients could have a very complicated effect on the evolution of a system. As the Brownian particle evolves in a potential it creates temperature gradients that effect the future diffusion of the particle, thus creating a rich interaction between the Brownian particle and it's environment. The dynamical behavior of the Brownian particle was fully described using numerical techniques, these numerical techniques were tested thoroughly using convergence tests and by a comparison to analytical solutions. Using our numerical techniques, we made concrete observations of the reverse Landauer blowtorch effect. We also saw that self-induced temperature gradients had an effect on the Kramers' rate and that they could cause a Brownian particle to go into a quasi-stable state where the particle becomes stuck in a particular well.


%\section{Future work}
%Our work is valid under certain assumptions, future work could build on our work could build on our results by relaxing these assumptions.
%
%\begin{itemize}
%    \item{Generalization to 2d and 3d systems}
%    \item{Inter-particle interactions}
%    \item{Flow of heat via convection as well as an extension to compressible fluids and gases}
%    \item{Implications to information theory}
%\end{itemize}
%
%\subsection{Generalization to 2d and 3d systems}
%Many systems of interest are multi-dimensional and cannot be approximated by a one-dimensional system \cite{KellerBustamante2000,Magnasco1994,Reimann2001,ChallisJack2014,M.W.Jack2016}. In particular, finding the efficiency of a molecular motor involves calculating the flow of energy from one degree of freedom to another \cite{M.W.Jack2016}. Furthermore, generalizing our system to more dimensions gives more directions in which heat can flow, therefore allowing us to conserve energy in a more general sense. The equations motion in multiple dimensions can be written as
%\begin{eqnarray}
%\mathbf{J}(\mathbf{x}, \, t) &=& - (P(\mathbf{x}, \, t) \nabla V(x) + k_B T(\mathbf{x}, \, t) \nabla P(\mathbf{x}, t)), \\
%\frac{\partial P(\mathbf{x}, \, t)}{\partial t} &=& -\nabla \cdot \mathbf{J(\mathbf{x}, \, t)}, \\
%\text{and} \nonumber \\
%\frac{\partial T(\mathbf{x}, \, t)}{\partial t} &=& -\kappa \mathbf{J(\mathbf{x}, \, t)} \cdot \nabla V(\mathbf{x}, \, t) + D \nabla^2 T(\mathbf{x}, \, t).
%\end{eqnarray}
%
%\subsection{Inter-particle interactions}
%Inter-particle interactions play a very important role in Brownian dynaimcs \cite{leibler1990physical,Leibler1993}, an example of this is the myosin protein in cells which is used in the contraction of muscles \cite{TyskaWarshaw2002}. Individually, myosin proteins are poor at carrying loads when compared to their kinesin couterparts \cite{TyskaWarshaw2002}, however their collective actions are very powerful due to the rich nature of their inter-motor interactions.
%
%The set of equations that we have been using do not apply to more than one Brownian particle without some considerable changes. This is due to very complicated inter-particle interactions that can occur when considering thermal fluctautions of the kind that we are focusing on. To visualize this intuitively, imagine two particles diffusing in their reaction coordinates, both subject to the same potential. Particle one may induce local temperature gradients that will affect particle two if particle two moves to where particle one created those temperature gradients. These interactions involve second-order statistics that are beyond the scope of this project.
%
%\subsection{Fluid dynamics} \label{fluidDynamics}
%The Brownian particles that we have been modelling have been suspended in a fluid. The dynamics of fluids present a formidable and exciting challenge to both physicists and mathematicians alike. As well as this there has been effort in the literature to combine the Smoluchowski equation with the Navier-Stokes equation, thus linking the world of Brownian dynamics with fluid dynamics \cite{Constantin2007}. The two physical phenomena that we neglect which could be treated with techniques from fluid dynamics are compressible fluids and heat convection through fluid flow. Despite the fact that we neglected these phenomena, our system is still physical and is self-consistent.
%
%In \autoref{thermodynamics}, in order to gain insight on the entropy we had to assume that we were dealing with an incompressible fluid. Without this assumption, we would not be able to guarantee that our equations of motion
%increase the entropy. Therefore, in order to obey the laws of thermodynamics in a compressible fluid, we would have to include extra terms in our equations of motion.
%
%The second important phenomena that can be described by fluid dynamics is fluid flow. This phenomenon is responsible for heat transport via convection, currently we only consider heat flow via diffusion, this means that we are tacitly assuming that the fluid in which our Brownian particle is suspended, is stationary. Including convection into our model would involve modifying the heat equation (equation \ref{eqn:TemperatureEvolution}) so that it reads
%
%\begin{equation}
%\frac{\partial T}{\partial t} + u(x) \frac{\partial T}{\partial x} = -\kappa J \partial_x V + D \frac{\partial^2 T}{\partial x^2} 
%\end{equation}
%Where $u(x)$ represents the flow of the fluid, one can imagine that such a flow would have an effect on the system by transporting the temperature gradients with the flow of the fluid. The next step in including fluid dynamics into the system would be to introduce the Navier-Stokes equations into our system, thus yielding three coupled equations to solve.
%
%\subsection{Information theory and computation with Brownian particles}
%The link between statistical mechanics and information theory has been established by Landauer \cite{Landauer1961}. In this paper Landauer considers bi-stable potentials similar to those discussed in ~\autoref{Kramers}. He then uses statistical mechanics to show that the erasure of information is associated with a generation of entropy, this therefore puts a theoretical lower bound on the amount of heat that a computer must produce in order to operate. Ref \cite{MyersCelebranoKrishnan2015} gives an experimental realization of information storage and retrieval using a levitating colloidal particle. Our model considers the flow of heat and its effect on the environment explicitly, therefore an obvious future project would be to reconsider Landauer's logic in the case of self-induced temperature gradients. By considering the effect that the Brownian particle has on the environment, one could possibly gain some considerable insight into the role of statistical mechanics in computation.


The equations that we have used are self consistent in that they describe a system that obeys the laws of thermodynamics. Despite this, we believe that due to assumptions made at the beginning of the thesis, the equations can only describe a limited number of systems actually realized in nature and in ~\autoref{outlook} we will discuss how the system could be generalized to incorporate physical phenomena from disciplines outside of statistical mechanics. Despite the fact that our system does not describe a large number of situations in nature, the lessons learned from studying our system will be extremely useful in modeling different situations. In particular, by observing self induced temperature gradients through numerical simulations we have been able to make concrete observations of the reverse Landauer blowtorch as well as the effect of thermal diffusivity and the specific heat of the environment on the Kramers' rate.  

Purely from a theoretical stand-point, the inclusion of self-induced temperature gradients better explains the flow of energy and entropy through a system. In ~\autoref{thermodynamics} we explained how the temperature of the environment is a crucial player in the calculation of the entropy and energy of a system.  In the case of deep wells, self-induced temperature gradients can actually cause the particle to stop moving with the force of the potential. This is an unexpected consequence of the reverse Landauer blowtorch and leads to some very interesting non-linear behavior. This is particularly interesting in Brownian motors, where the effect could lead to a stall in the motor, we speculate that biological systems will have mechanisms in place to avoid these stalls because a motor stall would mean that the motor would no longer be able to perform it's designated task.
%
%The courageous reader may want to use our model in the case of a compressible fluid, however in this case we can no longer guarantee that entropy will increase.