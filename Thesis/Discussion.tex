\epigraph{I say that the physics works by pixies, you say that the physics works by gnomes.}{\textbf{Associate Professor Colin Fox -- 2016}}

Brownian dynamics is an area of research that is of interest from the perspective of basic physics as well as applied physics. In this thesis we have discussed Brownian dynamics from a very basic point of view and we have focused in particular on the role of heat in the equations of motion. The equations that we have used are self consistent in that they describe a system that obeys the laws of thermodynamics. Despite this, we believe that due to assumptions made at the beginning of the thesis, the equations can only describe a limited number of systems actually realized in nature and in ~\autoref{outlook} we will discuss how the system could be generalized to incorporate physical phenomena from disciplines outside of statistical mechanics. Despite the fact that our system does not describe a large number of situations in nature, the lessons learned from studying our system will be extremely useful in modeling different situations. In particular, by observing self induced temperature gradients through numerical simulations we have been able to make concrete observations of the reverse Landauer blowtorch as well as the effect of thermal diffusivity and the specific heat of the environment on the Kramers' rate.  

Purely from a theoretical stand-point, the inclusion of self-induced temperature gradients better explains the flow of energy and entropy through a system. In ~\autoref{thermodynamics} we explained how the temperature of the environment is a crucial player in the calculation of the entropy and energy of a system. The fact that our system has a well defined entropy is a strong advantage and allows one to talk about information in a rigorous way \cite{Landauer1961,MyersCelebranoKrishnan2015}. Moving beyond the analytical solutions, we found that allowing self induced temperature gradients could have a very complicated effect on the evolution of a system. As the Brownian particle diffuses it creates temperature gradients that effect the future diffusion of the particle, thus creating a rich interaction between the Brownian particle and it's environment. In the case of deep wells, self-induced temperature gradients can actually cause the particle to stop moving with the force of the potential. This is an unexpected consequence of the reverse Landauer blowtorch and leads to some very interesting non-linear behavior. This is particularly interesting in Brownian motors, where the effect could lead to a stall in the motor, we speculate that biological systems will have mechanisms in place to avoid these stalls because a motor stall would mean that the motor would no longer be able to perform it's designated task.

The courageous reader may want to use our model in the case of a compressible fluid, however in this case we can no longer guarantee that entropy will increase.